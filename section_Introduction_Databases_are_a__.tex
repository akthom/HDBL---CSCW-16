\section{Introduction}

Databases are a common fixture of the contemporary workplace, and a key site of cooperative work. Yet, despite this omnipresence there has been relatively little research conducted by the CSCW community exploring how, exactly, cooperative work is done with and around databases;  how they structure the work that is or can be done with them (and vice versa); and how that work changes over time when the underlying relational model does not. 

We can imagine a few reasons for this lack of attention: databases (particularly the relational kind) have been around long enough to be assumed a “solved” problem; much of database work is done by people sitting alone in cubicles, giving it the appearance of independence rather than cooperativeness; and long term information system use is inherently difficult to study. Further, the end of the relational database has been predicted for decades (e.g. Atzeni et al., 2013), perhaps placing it in a sort of anticipatory blindspot: why study something that probably won’t be around in another five years? 

The object-oriented, XML or NoSQL solutions being touted today may well prove to be more effective solutions than the relational database. But, as a pragmatic field concerned with how people actually work as opposed to how people might work, we have to recognize that the relational database – particularly “desktop” databases of moderate size that aren’t necessarily stored on a server – still occupy an important middle-range niche. They are used by small groups of (often relation-savvy) users who want or need more structural complexity than flat-file spreadsheet programs such as Excel, and yet do not necessarily require enterprise-level systems. The study of other mundane technologies such as email, shared calendars, and spreadsheets has yielded interesting and helpful insights about how the use of these applications have evolved in use over time  \cite{bellotti2005quality}; \cite{palen1999social};  \cite{nardi1991twinkling}. This body of work reminds us that the innovative ways that people adapt, tailor and appropriate a widely used technology can be helpful for purposefully designing new information systems, or adding useful features to existing systems. Moderate size databases seem to  belong in Star's category of "boring things" \cite{star1999ethnography} -  mundane and often overlooked, but important elements that contribute to collaboratively creating effective infrastructures.  

In this paper, we want to emphasize that database use, maintenance and migration – particularly over the long-term – is indeed cooperative work, and that it is particularly influenced by the underlying relational data model. We focus on these activities in natural history museums, which are sites of collaborative scientific research. In the natural history setting databases not only function as infrastructure for information storage and retrieval, but also serve as an analytical tools for understanding collections of data in aggregate. 

By looking specifically at how the relational model of databases foster, and yet simultaneously constrain scientific collaborations we demonstrate their importance as a future unit of analysis for CSCW. Further, we show how the process of curating and controlling data models is inherently sociotechnical: In other words, the curation of data and the design of data models is a problem for sociotechnical scholars in CSCW, and as a result CSCW perspectives on these design problems can and should make data curation and data modeling more conducive to successful long-term collaboration. 

In the following sections we outline certain original design intentions of the relational model, and how it is that Codd's ingenious technical solution became a sociotechnical dilemma in practice. We then draw on Brand's work on how physical buildings adapt to a context of use, despite the limitations of their underlying architectural models. We use Brand's approach to describe the similar ways in which databases also adapt and evolve through their collaborative use, despite the rigidness of their underlying data models. We demonstrate this through a set of case studies in the field of natural history - including collaborations that depend on relational databases for zoology, paleontology, and microplanteology research.  

We specifically address three research questions: 

\begin{enumerate}
\item How do databases shape work practices, and vice versa?
\item How is relational database software adapted for use in contexts where there is no clearcut division between administrators and users?
\item What gaps exists between user needs of databases and the technical capabilities of current systems?
\end{enumerate}