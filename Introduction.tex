\section{Introduction}

Databases are a common fixture of the contemporary workplace, and a key site of cooperative work. Yet, despite their omnipresence there has been relatively little research conducted by the CSCW community exploring how, exactly, cooperative work is done with and around databases;  how their underlying data models structure the workplace practices of their users; and how that work changes over time when the underlying data model does not. 

We can imagine a few reasons for this lack of attention: databases (particularly the relational kind) have been around long enough to be assumed a “solved” problem; much of database work is done by people sitting alone in cubicles, giving it the appearance of independence rather than cooperativeness; and long term information system use is inherently difficult to study \cite{Pipek_2002}. Further, the end of the relational database has been predicted for decades (e.g. Atzeni et al., 2013), perhaps placing it in a sort of anticipatory blindspot: Why study something mundane now, that is not likely to be around in another five years? 

RDF stores, XML or NoSQL may well prove to be more effective solutions than the relational database. But, as a pragmatic field concerned with how people actually work as opposed to how people might work, we have to recognize that the relational database – particularly “desktop” databases not stored on a server – still occupies an important middle-range niche of workplace technologies. Relational databases are often used by small groups of users who want or need more structural complexity than flat-file spreadsheet programs such as Excel, and yet do not necessarily require enterprise-level systems. The study of other mundane technologies such as email, shared calendars, and spreadsheets has yielded interesting and helpful insights about how it is that ubiquitous applications have evolved in use over time  \cite{bellotti2005quality, palen1999social, nardi1991twinkling}. This body of work reminds us that the innovative ways that people adapt, tailor and appropriate a widely used technology can be helpful for purposefully designing new information systems, or adding useful features to existing systems. Moderate size databases seem to  belong to this category of technologies, what in Star's words were"boring things" \cite{star1999ethnography} -  mundane and often overlooked, but important elements that contribute to collaboratively creating effective infrastructures.  

In this paper, we want to emphasize that database use, maintenance and migration – particularly over the long-term – is indeed cooperative work. We focus on these activities in natural history museums as sites of collaborative scientific research. In the natural history context, databases not only function as infrastructure for information storage and retrieval, but also serve as an analytical tools for understanding collections of data in aggregate. 

By looking specifically at how the relational model of databases foster, and yet simultaneously constrain scientific collaborations we demonstrate their importance as a future unit of analysis for CSCW. Further, we show how the process of curating and controlling data models is inherently sociotechnical: In other words, the curation of data and the design of data models is a problem for sociotechnical scholars in CSCW, and as a result CSCW perspectives on these design problems can and should make data curation and data modeling more conducive to successful long-term collaboration. 

In the following sections we outline certain original design intentions of the relational model, and how it is that E.F. Codd's ingenious technical solution became, in practice, a sociotechnical dilemma. We then draw on Brand's work describing how physical buildings adapt to a context of use, despite the limitations of their underlying architectural models. We use Brand's approach to describe the similar ways in which databases also adapt and evolve through their collaborative use, despite the rigidness of their underlying data models. We demonstrate this through a set of case studies in the field of natural history - including collaborations that depend on relational databases for zoology, botany, and paleontology research.  

Our research question is: How does the structure, content and use of scientific databases change over longer periods of time. [please tidy up]
