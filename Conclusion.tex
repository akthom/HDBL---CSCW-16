\section{Conclusion & Future Work}
In looking at Natural History Museum databases, and how they have changed over time we have uncovered a number of interesting issues:
list here....

The study has highlighted a range of different kinds of cooperative work, including some that are on the edge of what some might even consider cooperation. We have classic coordination of volunteers, but also rather more attenuated and less visible collaborations. These include the periodic handover of a database to the next collections manager: a relatively rare and often problematic event in sharing expertise and experience. They also include the work of the collections manage seen as a provider of infrastructure to others where some of the cooperative work may be less visible to the beneficiaries of the infrastructure, and yet can involve many CSCW issues of awareness, coordination, etc.

 “There is no doubt whatever about the influence of architecture and structure upon human character and action. We make our buildings and afterwards they make us. They regulate the course of our lives”. A similar interaction happens between the design of the database and the people designing and modifying it.