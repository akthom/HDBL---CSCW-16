\section{Conclusion & Future Work}

In this paper we have looked at how natural history museum databases change over time and how that change affects their usability and use; in doing so, we have illustrated the specific, sociotechnical ways that databases and work with databases mutually constitute one another. Further we have shown

describes the long-term evolution of mundane artifacts, like databases, through a process of adaptation and learning 2. It develops a typology of databases, and discusses the changes in structure and use that occurs as a result of those changes; 3. It re-frames both data modeling and schema development as on-going (rather than initial) design tasks for CSCW.

Our findings show the importance of treating information modeling as an on-going design concern, as well as the need to develop tools that can assist users with the management and implementation of data models over time. Further, CSCW-concerned database designers should take note of the frequency that database users must export data to flat-file programs like Excel for data curation and normalization tasks. The existing ways in which flat files are used for data cleaning clearly point to functionalities that could be added to database software to support not just information storage and retrieval, but data curation as well.  For instance, two of the databases in this study were curated and migrated with the help of Open Refine\footnote{http://openrefine.org/}, a (now) open source data cleaning "power tool" that allows users to perform sophisticated data cleaning operations in a simple, Excel-like interface.  In short, Open Refine is the kind of tool that might be directly integrated with database programs. We also point to methodologies such as OntoClean, a formal methodology for analyzing information models and their consequences \cite{Guarino_2004} and tools that automate semantic encoding of spreadsheet data, such as OntoMaton \cite{Maguire_2012} as future directions for collaborative data work in this domain. 



\subsubsection{Design recommendations/future work?}

\textbf{design: }
- how do we design relational databases that support not just information storage and retrieval, but rather,  curatorial work and collections management? e.g. things like georeferencing; taxonomic referencing; clustering and batch editing as in Open Refine. (or: how can we merge open refine with specify?)
- Like Buneman, we see a definite need to support better provenance tracking in curated databases \cite{Buneman_2006}; however, we would argue that the actions tracked or bundled at at a higher level than "insert, delete, copy, and paste" actions -- while the core of data curatorial work may appear to a computer like a sequence of insertions, deletions and copies
- additionally: need to support visual rearrangement and browsing of tables and tuples; this might have SQL at its core but needs to be more user friendly. Bulk of HCI/UX studies have been about query construction - less has been written about user-friendliness of database GUIs, particularly from a curatorial perspective (e.g. someone doing quality assessment, editing, as opposed to just data entry, or just information retrieval
- need support for review of students'/volunteers' data entry  -- right now there's no great way to handle this.

\textbf{future work:}
- logical modeling as a design problem for CSCW
- in what ways do people come to know an information system? we need to study not just long lived databases but also people who work with databases for a longer period of time - people who gain almost tactile knowledge of information systems based on consistent use (as opposed to studies of how naive users use a database once).
- We'd like to study comprable organizations with relational databases.

Brand cites a quotation by Churchill: “There is no doubt whatever about the influence of architecture and structure upon human character and action. We make our buildings and afterwards they make us. They regulate the course of our lives”. A similar interaction happens between the design of the database and the people designing and modifying it.

