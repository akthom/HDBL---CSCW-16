\section{Conclusion & Future Work}


\subsubsection{Design recommendations/future work?}

\textbf{design: }
- how do we design relational databases that support not just information storage and retrieval, but rather,  curatorial work and collections management? e.g. things like georeferencing; taxonomic referencing; clustering and batch editing as in Open Refine. (or: how can we merge open refine with specify?)
- Like Buneman, we see a definite need to support better provenance tracking in curated databases \cite{Buneman_2006}; however, we would argue that the actions tracked or bundled at at a higher level than "insert, delete, copy, and paste" actions -- while the core of data curatorial work may appear to a computer like a sequence of insertions, deletions and copies
- additionally: need to support visual rearrangement and browsing of tables and tuples; this might have SQL at its core but needs to be more user friendly. Bulk of HCI/UX studies have been about query construction - less has been written about user-friendliness of database GUIs, particularly from a curatorial perspective (e.g. someone doing quality assessment, editing, as opposed to just data entry, or just information retrieval
- need support for review of students'/volunteers' data entry  -- right now there's no great way to handle this.

\textbf{future work:}
- logical modeling as a design problem for CSCW
- in what ways do people come to know an information system? we need to study not just long lived databases but also people who work with databases for a longer period of time - people who gain almost tactile knowledge of information systems based on consistent use (as opposed to studies of how naive users use a database once).
- We'd like to study comprable organizations with relational databases.

In looking at Natural History Museum databases, and how they have changed over time we have uncovered a number of interesting issues:
list here....

The study has highlighted a range of different kinds of cooperative work, including some that are on the edge of what some might even consider cooperation. We have classic coordination of volunteers, but also rather more attenuated and less visible collaborations. These include the periodic handover of a database to the next collections manager: a relatively rare and often problematic event in sharing expertise and experience. They also include the work of the collections manage seen as a provider of infrastructure to others where some of the cooperative work may be less visible to the beneficiaries of the infrastructure, and yet can involve many CSCW issues of awareness, coordination, etc.

 “There is no doubt whatever about the influence of architecture and structure upon human character and action. We make our buildings and afterwards they make us. They regulate the course of our lives”. A similar interaction happens between the design of the database and the people designing and modifying it.