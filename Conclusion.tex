\section{Conclusion & Future Work}

In this paper we have looked at how natural history museum databases change over time and how that change affects their usability and use; in doing so, we have illustrated the specific, sociotechnical ways that databases and work with databases mutually constitute one another. As Churchill (by way of Brand) says,
\begin{quote}
“There is no doubt whatsoever about the influence of architecture and structure upon human character and action. We make our buildings and afterwards they make us. They regulate the course of our lives”. 
\end{quote}
A similar interaction happens between the design of the database and the people designing and modifying it.

Our findings show the importance of treating information modeling as an on-going design concern, as well as the need to develop tools that can assist users with the management and implementation of data models over time. Further, CSCW-concerned database designers should take note of the frequency that database users must export data to flat-file programs like Excel for data curation and normalization tasks. The existing ways in which flat files are used for data cleaning clearly point to functionalities that could be added to database software to support not just information storage and retrieval, but data curation as well.  For instance, two of the databases in this study were curated and migrated with the help of Open Refine\footnote{http://openrefine.org/}, a (now) open source data cleaning "power tool" that allows users to perform sophisticated data cleaning operations in a simple, Excel-like interface.  In short, Open Refine is the kind of tool that might be directly integrated with database programs. We also point to methodologies such as OntoClean, a formal methodology for analyzing information models and their consequences \cite{Guarino_2004} and tools that automate semantic encoding of spreadsheet data, such as OntoMaton \cite{Maguire_2012} as future directions for collaborative data work in this domain. 
