\section{Discussion}
Comparisons with related studies
Our findings resonate with those of other researchers.
Voida et al's study of coordinators of volunteers at various non-profit organizations has interesting similarities and differences from our study of Natural History Musems. They found considerable use of other software (especially personal office applications such Excel and Outlook) to record information that might be expected to be more appropriately stored in a database. We also found considerable Excel use but not Outlook. The difference may be accounted for by one of the main activities of the volunteer coordinators being to communicate with volunteers. Where that communication is via email it is understandable that data may tend to accumulate in the email tool - particularly when that tool happens to work well enough for that data use. Problems of course arise when the tool is stretched to far to act as a data repository. This is less than ideal for a number of reasons, and yet it persists. Voida et al. found various reasons including low cost and familiarity (eliminating additional training needs). With fluid memberships, familiarity was also very important, as was flexibility and a good-enough multi-user functionality. Collections managers typically had much more database expertise and comfort, and yet at times they too used spreadsheets to store data. We believe this is an important point of comparison. It can be tempting to say that some people use spreadsheets to store data rather than a proper database because they don't know any better or lack the skills to develop and maintain a database. This was not the case for our collections managers and so we must seek other reasons for their use (at times) of spreadsheets.

A telling perspective from the NPO study is that “information management is not the real work of volunteer coordination; it is overhead.” This naturally affects the allocation of effort, resources and indeed enthusiasm. For collections managers, information management is far more central. But if we see similarities in activity between these different settings despite the contextual differences it requires us to seek better explanations of why people use the technologies the way that they do. From the NPO study: “In lieu of a system that can do everything, volunteer coordinators continually reconfigure their homebrew databases—swapping one system for another and hoping the new set of systems will help reduce overhead in managing information.”  and: “We heard over and over again that volunteer coordinators were in the process of migrating their data from one application to another.” These migrations seem more frequent and perhaps more ad hoc than in our natural history museum settings, but that migrations are a recurrent issue is important to bear in mind. If you are aware that your database is likely to be migrated a number of times in its life, this may have an impact in how you design and maintain it. The act of migration is often problematic. Inthe case of the NPOs: "Existing data either has to be ported—frequently necessitating manual re-entry of the data or selective copying and pasting—or abandoned. New systems rarely, if ever, encompass the same set of features or afford the same degree of flexibility as previous systems. Changes in the information managed by one application influence information management in others.” Despite better database skills and a closer alignment of database management with the 'real work' of the organization, certain similarities of re-entry and particularly of checking occurred in museums.

Another important piece of work that we want to use to compare our findings with involves a collaborative project between database researchers and biologists Jagadish et al. (2007). This uncovered a number of database usability issues that can arise when domain experts (but who are not necessarily  database experts) have to deal with the development and use of databases as part of their work. In general terms they note: “When we speak of usability, we mean much more than just the user interface, which is only a part of the usability equation. A more fundamental concern is that of the underlying architecture.” We too want to uncover the way that the underlying architecture affects use, and in particular changing use over time. As database expects they also note an irony in normalization. Codd proposed normalization as a way to protect end users from the underlying structure of the data, and make that data more robust over time and incremental changes. Jagadish note that “We break apart information during the database design phase such that everything is normalized—space efficient and updatable. However, the users will have to stitch the information back together to answer most of the real queries. The fundamental issue is that joins destroy the connections between information pertaining to the same real world entities and are nonintuitive to most normal users." As a result, databases may fail to be normalized for reasons of usability - or even deliberately de-normalized. [any examples, riffs on these ideas from our data?] Finally spreadsheets recur - but with a usability perspective on why: “While joins across multiple normalized tables may be difficult, people are certainly used to seeing data represented in simple two-dimensional tables. The popularity of the Excel spreadsheet as a data model speaks to this. For situations where data can be represented conveniently as a table, a tabular model is certainly appropriate."
