\section{Discussion}

Like buildings, databases adapt and change over time in response to their users -- and those constraints and changes at the logical level, in turn, can impact ways and styles of work.  Here we discuss in further depth the implications of this co-shaping of work for CSCW, particularly as it affects work and information systems in the very long term. We identify a number of implications not just for CSCW research and design, but also for that of data curation and digital preservation systems, and argue that a CSCW perspective could greatly improve the quality of the latter. [needs to be revised]

\subsection{Very temporally dispersed cooperative work}

The CSCW community has long studied contexts of great spatial or geographic dispersion; here we contribute to recent work exploring how CSCW is done across dispersed \textit{temporal} distances (e.g. \cite{Jackson_2011, Lindley_2015}). However, where several studies have explored how time matters in fast-paced, time-sensitive environments such as nursing \cite{sarcevic2009information, Reddy_2006} and paper mills \cite{auramaki1996paperwork}, ours contributes to an understanding of how time matters in extremely long-lived, if slower-paced environments: memory institutions. Museums must, by dint of their preservation mandate, take an extremely long view of work; as Bowker has noted, the process of databasing the world is a fundamentally a long-term one, and one that makes most of its progress through slow, steady curatorial work \cite{Bowker_2000}. However, this work isn't without interruption. Because databases have longer tenures at museums than their managers -- and longer lifespans than the software used to store them -- transitions between people and technology alike can be disruptive. Thus, for NHMs, the most critical point of information transfer is not at a shift change, as it is in medical work; rather it is the point at which one collections manager leaves, and the next comes in to take over care of both the collection and the collection databases. These are information handovers at the scale of generations.

Reframing this process as temporally dispersed cooperative work helps us identify areas in need of design interventions, particularly given that a specific technical artifact -- the database -- is being used for the handover, as well as being an artifact whose care is handed over. In our cases, we see that despite these being curated databases, they are not databases that particularly support curatorial work: they lack much of the functionality that collection managers need to re-control vocabularies after years of use (or misuse), or review data entered by volunteers, or reconcile their data with external sources, thus leading to the continued use of programs like Excel. Further, we see that many databases are simply not constructed with the expectation of periodic -- or even continual -- misuse, that eventually needs to be refactored.

A long-view of cooperative work shows us that there is a need to design relational databases that support not just information storage and retrieval, but rather, data curation as well: tools that allow data managers to assess changes to data and schemas that accrue over time, and make changes to the database's structure or contents as needed. Additionally, like \cite{Buneman_2006} and \cite{jagadish2007making}, we see a definite need to support better provenance tracking in curated databases; however our cases contribute to an understanding what provenance information is needed at a higher level than even the "coarse-grained" provenance they describe. Not only must users know what changes were made over time, but they must understand what social and technical constraints or concerns motivated changes. Further, our cases echo Voida, Harmon & Al Ani's finding that there's a need to not just track provenance within one system, but between data systems (e.g. between Excel and databases) \cite{voida2011homebrew}.

\subsection{Schemas, software, and the co-shaping of work}

One of the motivating ideas behind this work was that databases shape people just as much as people shape databases; that is, that databases can structure how people do work. In our cases, work with databases was co-shaped by people, data models, and software in several key ways: 
\begin{enumerate}
\item The kind and format of data that can be stored in a database are fundamentally shaped by not just the data model, but also the user’s ability to change that model. Where Codd may have intended that the logical schema be amended to accommodate new data, ad hoc workarounds (putting data into notes or remarks fields, co-opting fields, or exporting the database to a flat file for curation) are often used when users cannot change the schema due to other constraints (e.g. they don't have the necessary administrative permissions to change the schema, or they simply don't know how). These workarounds may work reasonably well at the time of their initial implementation, but can have unexpected effects when databases need to be migrated either to ward off obsolescence or meet evolving community needs (as illustrated by the problems encountered by Specify users in migrating co-opted fields). 
\item Even when databases are initially well-normalized and built “up to code”, complex relationships between tables can eventually become a barrier to use \cite{jagadish2007making}, – particularly if there is a mismatch between the original creator’s ability to manipulate the data model and the user’s or new custodian’s. In our cases, this barrier became most obvious during the “handover” phase, in which a new collections manager takes over custody of a database and must painstakingly reverse engineer its structure. 
\item That said, complex relationships can also facilitate use -- when they are being managed by someone who is able to act as a true administrator of the database. For instance, the Decapod Systematics Database described above features a fairly complex relational structure designed to track changes, as well as make visible the reasons for those changes. Thanks to several well-implemented web forms, these relationships do not need to be understood by the users of the database – only by the administrator -- and the functionality they afford are fundamental to what makes the database a valuable community resource.
\item Databases learn their “vernacular” from their “neighbors” just as buildings do; in our cases, “neighbors” include colleagues within the same research community who encourage one another to adopt similar designs for later integration, or simply share data schemas to save each other some design work. Our case’s move toward use of Specify and Arctos may also be viewed as a number of databases adopting a common style. 
\item Databases create a demand for administrators. As we reviewed at the beginning of this paper, Codd’s relational model is rooted in an assumed division of labor between administrators and users: users do data entry and retrieval, and administrators manage the mappings between the physical, logical, and conceptual levels of the database. However, in our cases, users and administrators are often one and the same. Though many of the databases observed in this study started out as custom-made systems created by someone straddling the user/administrator divide, their migration to systems like Specify or Arctos reestablishes users and administrators as separate roles. Thus, Specify and Arctos aren't "just" pieces of software -- they do very little that a system like MySQL or Access can't. Rather, they are sociotechnical bundles of software and work arrangements.
\end{enumerate}

[There's an esoteric point I keep wanting to make here - about how there might be weird consequences to splitting administrative duties in this way, because the collections manager is still in charge of the physical collection, and still in charge of the database - but I think I'm going to skip it for now.]

\subsubsection{The mundane, but meaningful: towards a typology of databases}

This study has looked at what might be classified medium-sized databases: databases that certainly don't fall under the umbrella of "big data" behemoths that requires petabytes of storage, yet still are complex or large enough to require some sort of structure beyond a flat file spreadsheet. We have further classified these databases according to the following types:

- Collections management databases: these are used to support the management of a physical specimen collection by organizing and preserving data about the specimens. They have typically been migrated from paper formats and are meant to last for the forseeable future. These are used for information storage and retrieval and data curation.
- Research databases: these are created specifically to support scientific projects and answer specific research questions. They are often integrative in nature, drawing from many different collections and sources of data. Taxonomic databases, which organize bibliographic references to species definitions, are a subset of these. These are used for information storage and retrieval, data curation, and analysis.
- Transactional databases: these are used for processing and tracking transactions such as loans, purchases and accessions over time. While they are used for information storage and retrieval, they are not used for curation.

Each of these types of databases has distinct roles and uses within a museum, and thus, distinct design needs and considerations. Echoing Voida, we suspect that there are numerous other databases of similar size beyond NHMs,  where people “need to manage information too complex for paper or personal office applications, but who cannot confront the overhead of using enterprise "'solutions.'" \cite{voida2011homebrew}. Further articulating different types of databases may help us create "solutions" that sit between supercomputers and spreadsheets.