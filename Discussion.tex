\section{Discussion}
Comparisons with related studies
Our findings resonate with those of other researchers.
Voida et al's study of coordinators of volunteers at various non-profit organizations has interesting similarities and differences from our study of Natural History Musems. They found considerable use of other software (especially personal office applications such Excel and Outlook) to record information that might be expected to be more appropriately stored in a database. We also found considerable Excel use but not Outlook. The difference may be accounted for by one of the main activities of the volunteer coordinators being to communicate with volunteers. Where that communication is via email it is understandable that data may tend to accumulate in the email tool - particularly when that tool happens to work well enough for that data use. Problems of course arise when the tool is stretched to far to act as a data repository. This is less than ideal for a number of reasons, and yet it persists. Voida et al. found various reasons including low cost and familiarity (eliminating additional training needs). With fluid memberships, familiarity was also very important, as was flexibility and a good-enough multi-user functionality. Collections managers typically had much more database expertise and comfort, and yet at times they too used spreadsheets to store data. We believe this is an important point of comparison. It can be tempting to say that some people use spreadsheets to store data rather than a proper database because they don't know any better or lack the skills to develop and maintain a database. This was not the case for our collections managers and so we must seek other reasons for their use (at times) of spreadsheets.

Clearly for processing certain kinds of result, spreadsheets are an entirely appropriate tool. It is not surprising that scientists would import data from a database into a spreadsheet in order to run certain calculations. But spreadsheets get used for many other purposes – just as email systems do [belotti]. Reasons include familiarity and facility. If you use spreadsheets a lot, then through inertia it is just there, already running on you computer. Also through regular use you learn how to be able to do many things with them, you are comfortable doing them, and you may also be more comfortable innovating and tinkering with them.  Consequently spreadsheets may be a convenient location for both mundane and more experimental data use. These include data management activities such as data cleaning and consistency checking.  Other reasons include the relative visibility (and comforting familiarity) of the tabular representation (Jagadish) and the ease of checking the consequences of actions (Nardi). As a conceptually comfortable data cache, there can be relatively few concerns about spreadsheet use. However, as spreadsheets start to become slightly longer term repositories of data, or serve as a resource parallel to but unsynchronized with the database but  we might worry more. A buildings analogy might be a room that 'temporarily' becomes an additional storage space, but then never reverts to its main use.

A telling perspective from the NPO study is that “information management is not the real work of volunteer coordination; it is overhead.” This naturally affects the allocation of effort, resources and indeed enthusiasm. For collections managers, information management is far more central. But if we see similarities in activity between these different settings despite the contextual differences it requires us to seek better explanations of why people use the technologies the way that they do. From the NPO study: “In lieu of a system that can do everything, volunteer coordinators continually reconfigure their homebrew databases—swapping one system for another and hoping the new set of systems will help reduce overhead in managing information.”  and: “We heard over and over again that volunteer coordinators were in the process of migrating their data from one application to another.” These migrations seem more frequent and perhaps more ad hoc than in our natural history museum settings, but that migrations are a recurrent issue is important to bear in mind. If you are aware that your database is likely to be migrated a number of times in its life, this may have an impact in how you design and maintain it. The act of migration is often problematic. Inthe case of the NPOs: "Existing data either has to be ported—frequently necessitating manual re-entry of the data or selective copying and pasting—or abandoned. New systems rarely, if ever, encompass the same set of features or afford the same degree of flexibility as previous systems. Changes in the information managed by one application influence information management in others.” Despite better database skills and a closer alignment of database management with the 'real work' of the organization, certain similarities of re-entry and particularly of checking occurred in museums.

Another important piece of work that we want to use to compare our findings with involves a collaborative project between database researchers and biologists Jagadish et al. (2007). This uncovered a number of database usability issues that can arise when domain experts (but who are not necessarily  database experts) have to deal with the development and use of databases as part of their work. In general terms they note: “When we speak of usability, we mean much more than just the user interface, which is only a part of the usability equation. A more fundamental concern is that of the underlying architecture.” We too want to uncover the way that the underlying architecture affects use, and in particular changing use over time. As database expects they also note an irony in normalization. Codd proposed normalization as a way to protect end users from the underlying structure of the data, and make that data more robust over time and incremental changes. Jagadish note that “We break apart information during the database design phase such that everything is normalized—space efficient and updatable. However, the users will have to stitch the information back together to answer most of the real queries. The fundamental issue is that joins destroy the connections between information pertaining to the same real world entities and are nonintuitive to most normal users." As a result, databases may fail to be normalized for reasons of usability - or even deliberately de-normalized. [any examples, riffs on these ideas from our data?] Finally spreadsheets recur - but with a usability perspective on why: “While joins across multiple normalized tables may be difficult, people are certainly used to seeing data represented in simple two-dimensional tables. The popularity of the Excel spreadsheet as a data model speaks to this. For situations where data can be represented conveniently as a table, a tabular model is certainly appropriate."

Time, rhythms and handovers
A consideration of database use, change and maintenance over long periods of time has numerous connnections to CSCW. Handovers and handoffs are a recurrent theme in CSCW in a variety of settings ranging from medicine to paper mills [eg, Saracevic 2009, Auramäki 1996]. In the case of NHM databases a critical issue for a database to be successfully long-lived is the handover from one database manager to another. These occur at much longer time periods than the shift-level handover, (from years to decades) and so are likely to a far less practiced or routinized skill. Ideally a handover involves a period of time for face to face meetings, even an apprenticing into the role for the newcomer. But it can also be more indirect, involving the use of documents and a degree of reverse engineering. In problematic cases it is a kind of CSCW where there is far less collaboration than is desired, and certain computational resources compensate for the lack of collaboration. Both handovers and migrations have something of a rhythmic aspect, but they are are more like the bursty rhythms of Jackson et al. [2011]

Infrastructure, collaboration and levels of visibility
Star and Strauss (1999) developed a framework for analyzing invisible work in CSCW systems. They emphasize that “No work is inherently either visible or invisible.” Their examples are of clearly disempowered people. Star (1999) notes how we can surface invisible work while looking at an infrastructure, and gives the example of the work of secretaries in a research lab. By contrast NHM data managers are hardly invisible, but they and their work can at times be overlooked. Indeed when we consider the database as infrastructure for other work that uses the database, then its development and maintenance can be invisible, at times desirably or even definitionally so. As Star (1999) says of infrstructure “It is by definition invisible, part of the background for other kinds of work” 

The work of the collections managers may not be as invisible as the examples given by Star and Strauss. But they do have certain aspects of lowered visibility. This seems to relate to their role in maintaining a robust scientific infrastructure. Consider a very simple, and subjective definition of infrastructure: “infrastructure is someone else’s problem” In this egocentric framing, you would consider as infrastructure those other things that help you do the things you do – but that you don’t have to worry about too much because somebody else is worrying about them on your behalf. Any one of those people view yet other things as a part of their infrastructure in the same subjective way – and it is quite possible that some of what you do is part of their infrastructure even as some of what they do is part of yours. In this view of infrastructure, its invisibility and the invisibility of the work to make it work is a highly desirable feature. Invisibility means you don’t have to worry about it – except maybe when the infrastructure fails, becomes visible, you do have to worry about it and it (temporarily) become not-infrastructure until it gets fixed and fades back into invisibility.

It may be that some people working on such infrastructures take a professional pride in rendering their work somewhat invisible – since that means that the infrastructure they provide ‘just works’ . It can facilitate collaboration without being perceived as collaborative by the recipients. Such infrastructure providers may coordinate their work around the beneficiaries of the infrastructure that they provide. They may use a variety of activities and processes that we consider core to CSCW such as awareness, workflow coordination, articulation work etc – but the beneficiaries may not even be aware of the degree that they are being collaborated with – unless there is an infrastructure breakdown. As Star reminds us, infrastructure becomes visible upon breakdown. What we would add is that breakdown is often when collaboration also increases, - or maybe the infrastructure-enabling collaboration suddenly becomes more visible to some of the people involved. For example we may only talk to our cell phone providers when there is a problem.

The middling and the mundane, but meaningful
Our focus is on long-lived scientific databases in natural history museums where new entries are added, and certain entries and values may be revised over time. The databases are used by scientists and are expected to continue to be used for years, decades and centuries – just as their precursors have. These are not particularly large databases - they would not count as "big data" - nor do they contain rapidly changing transactional data. They are not databases whose contents are now fixed and are not currently being used, but need to be archived in case they are  needed in the future, as explored by Olson (2010). This last case of archiving is less like one of Brand’s learning buildings and more like an historic monument with a restrictive preservation order applied to it.

Although not large from the perspective of giant astronomy projects or corporate examples, they are  significant and are not trivially small. Many can reside in current powerful laptop computers, even it it would be better to have them on servers. They are unlikely to receive the attention of database researchers (Jagadish is a commendable exception) who are likely to be working on the challenges of much greater scale, levels of use and rates of change. But improving their usability for those who create, update and use them, and knowing the degree of difference (if any) between users and maintainers is important. In particular the collaborations around their use and particularly around their maintenance over decades (and, we can hop, centuries) is an important challenge where insights from CSCW can play a vital role.

We suspect that there are similar database uses and needs outside Natural History Museums, and not just in other scientific settings. Inspired by Voida et al we think there may be  many other domains ( where people “need to manage information too complex for paper or personal office applications, but who cannot confront the overhead of using enterprise “solutions.””. We would go further and note that with increasingly powerful machines there is an opportunity and even an expectation that people will be managing ‘moderate sized’ datasets. These are typically too large or unwieldy for paper or personal office applications, but look like very modest datasets from the perspectives of many systems administrators or especially database researchers. Our laptops and PCs may run them well, but we may not. The tools, features and interfaces needed for moderate databases are different from those for large ones. As Jagadish et al. remind us, too much power and many choices can be a significant disadvantage in moderate database usability.


