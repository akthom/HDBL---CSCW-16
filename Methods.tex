\section{Methods}

\subsection{Research Design}

This study seeks to understand the use of databases through time. In describing the problem of evaluating built environments, Duffy notes that "The unit of analysis isn't the building, it's the use of the building through time. Time is the essence of the real design problem." \cite{duffy1990measuring}. Similarly, the real design problem facing relational data models isn't in designing for initial use, but rather, designing for on-going and changing use. 

Therefore, to understand databases' evolution and use through time, we developed a multi-case study of databases that have been designed and used in natural history research over long periods of time. Each case (n=9) represents a different department within a different museum; and each database used within that department represents a different embedded sub-case (n=36; Table 1 & Appendix A). Thus, our unit of analysis is not the department or the people that work in that department, but instead the databases designed and used by that department. In past work (anonymized for review), we have sought to "interview" databases primarily through a trace ethnographic approach \cite{Geiger_2011}; here, however, we root our analysis in interviews with the databases' primary administrators, and then supplement that data with exploration of the database's structure. 

We discussed three distinct phases of a database's evolution with our participants: 
\begin{enumerate}
\item The database's state and format at the time that a participant began working at the museum (as well as his or her knowledge of any prior versions); 
\item The database's present state and format, and; 
\item Its future or anticipated state. 
\end{enumerate}
By gathering data on these three different states, we could  compare databases across time, departments, and institutions - seeking patterns in the way that content is normalized, migrated, and occasionally transferred between different database designs. 

An initial group of participants of five participants were selected based on their known use, management or design of natural history databases; an additional four were selected through snowball sampling. In total, 9 collections managers, researchers and curators at a range of NHMs and museum departments were enrolled in this study. While we sought representation across different types of institutions involved in natural history research (e.g. both universities and independent museums), different research departments within natural history museums, and different levels of technical training in our participants, we did not seek a statistically representative sample of the databases or their users in NHMs; the field is simply too broad for this approach to be feasible. Instead, our multi-case approach treats each case as a replication of an experiment and seeks to identify common phenomena and patterns between those cases \cite{yin2013case}. 

\subsection{Data Collection}

For each case, we undertook an initial demographic survey of the database administrator (typically a collections manager). We then conducted a semi-structured interview with this individual. The interviews focused on the participant's knowledge of a database's history, users of the database, changes in the use of the databases over time, and steps that the collections manager had taken to migrate, curate or otherwise manage the databases during their tenure at the museum. The interview protocol was developed, in part, by our examination of publicly available versions of each database. Where possible, we also obtained copies of the database schemas from our participants. Interviews lasted 45-75 minutes, and 60 minutes on average. Data collection began in Spring 2014 and concluded in Spring 2015; those participants interviewed in early 2014 (n=5) were contacted for a follow up interviews in 2015, primarily to assess the level of progress they had been able to make (or not) in migrating different databases. 

Interviews were transcribed and then coded for themes related to migration, evolution or change over time, and the impact of these changes on work practices. These themes, as well as documentation and database schemas collected from participants were then used to create short case reports describing the experience of each database administrator, and the history and evolution of each database with which he or she worked. Transcirpts and case reports were shared amongst all three authors, and discussed and reanalyzed to draw out relevant findings. In the following section we describe the results of this work.