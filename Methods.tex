\section{Methods}

\subsection{Research Design}
\textit{"The unit of analysis isn't the building, it's the use of the building through time. Time is the essence of the real design problem."}\cite{duffy1990measuring}

Similar in vein to Duffy's remarks above, this study seeks to understand the use of databases through time. The real design problem facing relational data models isn't in designing for initial use, but rather, in designing for on-going, changing use.  Thus, we need to study systems over time to better understand how use changes, how logical models changes reflect those changes is use, and vice versa.  Thus, our study questions are:
\begin{enumerate}
\item In NHMs, how do collections managers manage, curate and migrate their databases over time?
\end{enumerate}
\begin{enumerate}
\item In NHM databases, how do logical schemas change over time?
\end{enumerate}
\begin{enumerate}
\item In NHM databases, how do logical schemas impact work, and vice versa?
\end{enumerate}
\begin{enumerate}
\item In NHMs, how do changes in databases impact collaboration?
\end{enumerate}

And our propositions include:
\begin{enumerate}
\item Like buildings, databases learn -- that is, their structure changes in response to their users, their data and the software used to house the database
\end{enumerate}
\begin{enumerate}
\item NHM CMs may not have a large amount of formal training in database design or management, but they do consider database management and maintenance an important part of curatorial work ((Marty 2005); (Marty 2006))
\end{enumerate}
\begin{enumerate}
\item Databases will tend to become denormalized over time, and ad hoc work arounds may be used to make them work.
\end{enumerate}
\begin{enumerate}
\item This denormalization can have a long term detrimental effect on data quality, but may ease collaboration, at least in the short term
\end{enumerate}
\begin{enumerate}
\item Denormalization over time is ironically at least partially a side effect of the relational model, which aims to focus users on relationships rather than relations (in mathematical sense). In practice, the separation between relations and relationships is predicated on an administrator vs. user role distinction. However, in many places of cooperative work, those roles are blurred
\end{enumerate}
\begin{enumerate}
\item So, to summarize, this research will help us begin to understand how relational databases learn in a context of use in which there is no clear cut distinction between administrators and users? That is, what work needs to be done to make these designs "work"?
\end{enumerate}

Therefore, we developed a multi-case study of databases that have been designed and used in natural history research over time. Each case (n= 9) represents a different department within a different museum; however, our unit of analysis is not the department (or the collection manager working in a department), but rather, an individual database designed and used by that department (n = approx 36). Three units of observation were discussed with our participants: the database's state and format at the time that the participant began working at the museum (as well as her knowledge of any prior formats); the database's present state and format; and its future or anticipated state. This allows us to compare different databases across time, across departments, and across institutions - seeking patterns in the way that content is normalized, migrated, and occasionally transferred between different database designs. 

We intentionally did not seek to interview a statistically significant sample of the natural history research community; it is too broad and too diverse for this to be feasible.  Rather, we sought research participants from a range of disciplines and institutions, and hoped to find replications of phenomena across different fields, scales, and levels of experience. Participants were purposefully selected based on their use, management or design of natural history databases. In total, 9 collections managers, researchers and curators at a range of NHMs and museum departments were contacted in total (see table X below.) 

\subsection{Data Collection}
For each case, data were collected through an initial demographic survey of the data base administrator and a semi-structured interview.  Interviews concentrated on asking participants about the history of their databases, changes in the databases’ use over time, users of the database, and steps they had taken to migrate, curate or otherwise manage the databases during their tenure at the museum. The interview protocol was developed through analysis of publicly available versions of these databases, or prior knowledge of the databases’ technical structure. In some cases, we asked researchers for copies of the databases they work with. Interviews lasted 60 minutes on average.  Data collection began in Spring of 2014 and concluded in Spring of 2015; participants interviewed in early 2014 (n=5) were contacted for a follow up interviews in 2015, primarily to assess the level of progress they had been able to make (or not) in migrating their databases.

Interviews were transcribed and partially coded for thematic XXXX by XXX. These themes, as well as documentation and database schemas collected from participants were then used to create short case reports describing the experience of each database administrator, as well as  the history and evolution of each database she worked with.  

Our analysis below draws upon the themes ... 
