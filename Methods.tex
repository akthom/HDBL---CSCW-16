\section{Methods}

\subsection{Research Design}

This study seeks to understand the use of databases through time. in describing the problem of evaluating built environments, Duffy notes that "The unit of analysis isn't the building, it's the use of the building through time. Time is the essence of the real design problem." \cite{duffy1990measuring}. We believe that the real design problem facing relational data models isn't in designing for initial use, but rather, in designing for on-going, changing use. This is supported by previous studies that note, “The design of a database is almost impossible to get right on the first try. The best one can hope for is to build something that is workable and that can be extended without too much disruption to the applications that were written for the initial design.” \cite{buneman2008curated}

We developed a multi-case study of databases that have been designed and used in natural history research over time. Each case (n=9) represents a different department within a different museum; however, our unit of analysis is not the department or the collection manager working in a department, but rather, an individual database designed and used by that department (n = approx 36). Three phases of a database's evolution were discussed with our participants: the database's state and format at the time that the participant began working at the museum (as well as her knowledge of any prior versions); the database's present state and format; and its future or anticipated state. This allows us to compare databases across time, departments, and institutions - seeking patterns in the way that content is normalized, migrated, and occasionally transferred between different database designs. 

We intentionally did not seek to interview a statistically significant sample of the natural history research community; it is too broad and too diverse for this to be feasible. Rather, we sought research participants from a range of disciplines and institutions, and hoped to find replications of phenomena across different fields, scales, and levels of experience. Participants were purposefully selected based on their use, management or design of natural history databases. In total, 9 collections managers, researchers and curators at a range of NHMs and museum departments were contacted in total (see Appendix A).

\subsection{Data Collection}

For each case, we undertook an initial demographic survey of the database administrator (usually a collections manager). We then conducted semi-structured interview, which concentrated on the history of a database, changes in the databases’ use over time, users of the database, and steps that the collection manager had taken to migrate, curate or otherwise manage the databases during their tenure at the museum. The interview protocol was developed through analysis of publicly available versions of these databases, or prior knowledge of the databases’ technical structure. In some cases, we asked researchers for copies of the databases they work with. Interviews lasted 60 minutes on average. Data collection began in Spring 2014 and concluded in Spring 2015; those participants interviewed in early 2014 (n=5) were contacted for a follow up interviews in 2015, primarily to assess the level of progress they had been able to make (or not) in migrating different databases.

Interviews were transcribed and partially coded for themes related to migration, evolution or change over time, and the impact of these changes work practices. These themes, as well as documentation and database schemas collected from participants were then used to create short case reports describing the experience of each collections manager, as well as the history and evolution of each database she worked with. 

Our analysis below draws from these reports, and is organized around key themes. We further draw from Brand in our analysis, describing ways in which databases exhibit stylistic, regional vernaculars; grow beyond or in response to their pre-existing structure; and the steps participants take to bring their systems up to various regulations and "building codes."\