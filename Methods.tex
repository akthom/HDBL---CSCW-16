Because our research seeks to understand how and why databases are being used over time, and because prior works shows that the phenomenon of specific database use is difficult to separate from its idiosyncratic context, we take a multiple case study approach (after Yin, 2009, and others? I don’t know how to frame this).   We intentionally did not seek to interview a statistically significant sample of the NHM community; it is too broad and too diverse for this to be feasible.  Rather, we sought to interview individuals from a range of disciplines and institutions, and hoped to find replications of phenomena across different fields, scales, and levels of experience.

Participants [cases?] were selected based on their use, management or design of natural history databases. We consider each collection (both physical and digital) that she or he curates a separate case, and each database maintained as part of that collection an embedded subcase (Table 1).   9 collections managers, researchers and curators at a range of NHMs and museum departments were contacted in total, representing 9 different cases.   We note that 3 of these cases were collected from the same institution (CU Boulder), but represent different departments and scientific disciplines, and indeed, use different database systems.