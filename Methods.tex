\section{Methods}

\subsection{Research Design}
\textit{"The unit of analysis isn't the building, it's the use of the building through time. Time is the essence of the real design problem."}\cite{duffy1990measuring}

Similar in vein to Duffy's remarks above, this study seeks to understand the use of databases through time. The real design problem that is facing relational data models -- particularly in a preservation or curatorial capacity -- is that of designing a database that will sustain over time, and through  

Therefore, we developed a multi-case study of databases that have been designed, and used in natural history research over time. Each case (n= 8) represents a different department within a different museum. And each unit of analysis is a database designed and used by that department (n = 37). We account for changes over time through a unit of observation, that is what is actually observed, analyzed, and discussed with our participants, that represents three unique stages of a database's existence: how it arrived at (or was originally designed) at the museum, it's present state, and its future or anticipated state. This allows us to compare different databases across time, across departments, and across institutions - seeking patterns in the way that content is normalized, migrated, and occasionally transferred between different database designs. 

We intentionally did not seek to interview a statistically significant sample of the natural history research community; it is too broad and too diverse for this to be feasible.  Rather, we sought research participants from a range of disciplines and institutions, and hoped to find replications of phenomena across different fields, scales, and levels of experience. Participants were purposefully selected based on their use, management or design of natural history databases. In total, 9 collections managers, researchers and curators at a range of NHMs and museum departments were contacted in total (see table X below.) 

\subsection{Data Collection}
For each case, data were collected by an initial demographic survey of the data base administrator and a semi-structured interview.  Interviews concentrated on asking participants about the history of their databases, changes in the databases’ use over time, users of the database, and steps they had taken to migrate, curate or otherwise manage the databases during their tenure at the museum. The interview protocol was developed through analysis of publicly available versions of these databases, or prior knowledge of the databases’ technical structure. In some cases, we asked researchers for copies of the databases they work with. Interviews lasted 60 minutes on average.  Data collection began in Spring of 2014 and concluded in Spring of 2015; participants interviewed in early 2014 (n=5) were contacted for a follow up interviews in 2015, primarily to assess the level of progress they had been able to make (or not) in migrating their databases.

Interviews were transcribed and partially coded for thematic XXXX by XXX. These themes, as well as documentation and database schemas collected from participants were then used to create short case reports describing the experience of each database administrator, as well as  the history and evolution of each database she worked with.  

[Any external validity? Did you show these to participants? Solicit feedback from anyone as to whether the observations were or were not valid? ]

Our analysis below draws upon the themes ... 
