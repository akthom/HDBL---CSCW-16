In this paper, we describe the way that databases evolve through different contexts of use and collaboration. We draw on Brand's idea that buildings, even given their architectural constraints, \emph{learn} through adaptation to their inhabitants, modification to their changing surroundings, and by rehabilitation of both their inner and outer appearances. Databases, even given the constraints of their underlying logical models, also \emph{learn} through normalization and migration of their contents, and by their integration with other information systems. This process is demonstrated through a series of case studies from natural history museums, which develop databases to preserve data, and to structure collaborations around that data. This research contributes to understanding long-term methods of cooperative work through collaborative artifacts, and reframes information modeling as an on-going (rather than initial) design task for CSCW.