In this paper we describe the way that relational databases change through different contexts of use, and the impact that these changes have on cooperative work. We draw on Brand's idea that buildings, even given their architectural constraints, \emph{learn} through adaptation to their inhabitants, modification to their changing surroundings, and rehabilitation of both their inner and outer appearances. Databases, even given the constraints of their underlying logical models, also \emph{learn} through normalization and migration of their contents, and through integration with other information systems. This process of adaptation over time is demonstrated through a series of case studies from natural history museums, which develop databases to both preserve data and structure collaborations around that data. This research makes three contributions to CSCW: 1. It describes the long-term evolution of mundane artifacts, like databases, through a process of adaptation and learning 2. It develops a typology of database roles, and discusses the co-shaping of structure and use that occurs across those different roles; 3. It re-frames information modeling as an on-going (rather than initial) design task for CSCW.