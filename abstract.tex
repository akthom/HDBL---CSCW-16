In this paper, we study databases by bringing E. F. Codd in conversation with Stewart Brand: looking at databases and the logical schemas that underlie them in the same way that Brand looked at buildings, and how they change (or “learn” in Brand’s words) in their form, contents and use over time.  We do this through a multiple case study of natural history databases over time, a genre of long-lived, (typically) relational databases which bridge the divide between databases as a data preservation medium, and databases as tools of research.  Just as physical buildings “learn” from their changing contexts of use – within the constraints of their site and structure – so too do these databases learn from their users, the scientific domains and workplaces in which they are implemented, and, perhaps most importantly, the logical constraints of their underlying data models.  This research contributes to our understanding of rhythms of cooperative work, and reframes information modeling as an on-going (rather than initial) design task with implications for data curation and CSCW.