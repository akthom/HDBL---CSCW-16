\section{Theoretical Background}
Most contemporary databases follow E. F. Codd relational data model, which was meant to address problems of “shared” data banks, by  “[protecting] users from having to know how data is organized” in a computer system (pp. 377, \cite{Codd_1970}. Moving away from the tree-structures or network models of data popular in the 1970's, Codd conceived of a set theoretic approach to the representing data by separating a system's architecture into three distinct levels: 1. Physical levels describe how data are stored; 2. Logical levels describe what data are stored, and; 3. Views levels hide details of the physical and logical levels, and can be tailored to specific end user query needs. In hiding these details, Codd's relational model emphasizes users, as he wrote “there is one consideration of absolutely paramount importance – and that is the convenience of the majority of users” (1971, pp.2; emphasis his). 

In Codd's  model, the division of a database's architecture also results in a division of labor between database designers and administrators at the logical level, and end-users at the views level. In practice, these divisions aren't based on rigid types, but rather roles that different people play in interacting with a database at different points in time. An individual might help design and administer a database that she also uses as a part of her own research. This is especially common in information-rich-but-personnel-poor workplaces, where desktop applications such as Microsoft Access and FileMaker Pro reduce the burden of having to learn set-theory just to create a relational database. Many CSCW and HCI studies have described the problematic nature of these collapsed database roles, including: Batra's work on error types in database design (1993) which are described in terms of  four 'gulfs' that emerge between context of use and design; Olson's description of the organizational super-administrator (2010), and ; Dourish and Edwards design of a toolkit to reduce the burden of "locking" style commitments in software code or design constraints of database schema (2000). Our work focuses on how the division of views and logical levels of database architecture [..Finish sentence with example from our work..] 

Additionaly, many of the textbooks and examples from this literature rely upon a transactional use case - such as a parts database in a warehouse. While the transactional use case is a helpful pedagogical tool, it also  ignores the more complex uses of relational databases, such as those in a research setting where they are part infrastructure for information storage and retrieval, and part artifact in which research methods and worldviews are negotiated into being, and crystallized into canon (e.g. Bietz and Lee, 2009; Bowker, 2000, Jagadish et al., 2007.)  

\begin{itemize}
\item Lots of discussion of need for data independence between physical/logical levels – what about logical/conceptual? How do those interact?
\item And what about logical structure and day-to-day work practices? How do those interact?
\item Note related work in HCI/CSCW - Batra 1993; Topi & Ramish 2003; more
\end{itemize}

\subsection{Brand}

\begin{itemize}
\item Brand gives us a framework to start answering those questions
\item	it's unintuitive that a building learns from its inhabitants and vice versa’ Equally unintuitive is that conceptual and logical levels learn form one another (e.g. that databases and people learn from each other)
\item Brand’s 6 S’s as they translate to databases
\item Include figure showing table growth over time
\end{itemize}
Brand’s analysis of buildings [ref] highlights a number of important issues that arise when taking a longer term perspective. He contrasts the initial appearance of a building and the architect’s vision (often the major criteria used in determining awards) with the actual use experience over time of the people living in it. He continues this analysis by exploring use over years, decades and even centuries noting how the requirements of the inhabitants of the building change and so they change the structure and the meaning of parts of the building, adding, demolishing and modifying. Over longer time periods, a building may have different inhabitants and be used for changing purposes.   Brand’s ideas have been found useful in the context of an HCI perspective on ubiquitous technologies for domestic environments (Rodden & Benford 2003). We believe that his analysis of buildings can serve as a productive metaphor for exploring the way that databases change over similarly long time periods (Thomer & Twidale 2014)  
Change occurs at different rates, and Brand proposed six layers to describe these rates. [expand here or later]  
\subsection{Related Work in CSCW}
Although there was a consideration of databases in the early days of CSCW as a research field [Mariani & Rodden] \cite{mariani1991impact}
relatively little recent work explicitly addresses them. By contrast there is a growing body of work on infrastructures and collaborative aspects of scientific work [e.g refs? what?]. Some of this also focuses  on the importance of taking a long term perspective on the scientific process and the informational resources that support it [eg. Karasti et al.; Ribes]. However most of this work takes a higher level organizational perspective, considering issues such as governance and incentives. In this study we choose to look at the databases as the unit of analysis and what we can learn from they way that they change over time.  Voida et al. did a study at a similar level to ours, but in a somewhat different domain:  the coordinators of volunteers at various non-profit organizations (NPOs). NPOs and Natural History Museums do have certain things in common; both suffer from a lack of funding, and  funding for informational infrastructure can be particularly difficult to justify. However there are also important differences. We will compare our findings with those of Voida et al. in later sections. 

\subsection{The nature of natural history work}

We have selected natural history museums as the focus for this case study because they are an excellent "model organism" for [the kind of database work we will have presumably described above]. NHM work -- particularly, that done by collections managers (CMs) -- is deeply technosocial: it involves mediating between physical specimen collections (these can range from fossils to sea shells to whole organisms preserved in alcohol, depending on the discipline), paper records, relational databases and anticipated future platforms; and between volunteers, curators\footnote{we note that in NHMs, the "curator" tends to do less curation in the sense of physically caring for specimens and their data, but rather functions more like a researcher administrator: writing papers and grants, and making policy decisions. Collections managers are typically in charge of the the actual day-to-day maintenance of collections and databases.}, IT staff, visiting researchers, and other departments. Thus, any given day’s work will involve some combination of wrangling both people and information structures. Indeed, each CM we spoke with for this research spends anywhere from 40-80\% of her time working with databases (typically closer to the latter) or supervising/checking the work of volunteers working with databases.

The data in electronic collections databases is rarely created \textit{de novo}; rather, these databases are better viewed as the latest manifestation of a hundreds-year-old tradition of NHM record keeping. The data they contain have been "digitized" (transcribed) from paper ledgers, card catalogs, and individual labels. NHM's have been somewhat early adopters of databases technology, and many of the existing electronic catalogs were originally created via punch card technology in the 1970s and 80s; in the 90s many of these systems were migrated to custom built, but proprietary, systems such as dBase, Paradox and Access; and since the early 200s, more and more museums have been using "off-the-shelf" databases with pre-defined data schemas designed specifically for NHMs, such as Arctos, Specify and KE EMu. The move to community-developed databases such as Arctos and Specify is paralleled by a move toward community-developed data standards and schemas.  However, while many collections include similar information in their records – for instance, a catalog number, a specimen’s scientific name, a description of the locality from which it was collected – the way in which this data is recorded is often quite idiosyncratic to an institution’s local cataloging rules and practices, as well as to the domain of study; as a result, collections management databases often have correspondingly idiosyncratic legacy data structures.

Ideally, collections databases would act as both a finding aid for the physical collection, and a research dataset in and of themselves (see \cite{Chapman2005} for a broad review of research uses): listing the name and catalog number of each specimen, as well as additional contextualizing data such as the name of its original collector and the location from which it was collected. However, these databases are partial representations of the collection -- and, as Bowker notes, the natural world \cite{Bowker_2000} -- at best. The vast majority of museum collections are \textit{not} fully cataloged; and many of the catalogs that do exists are still in need of digitization \cite{Beaman_2012}. Of the estimated 1.2 billion specimens worldwide, only [number] 9\% have been cataloged, let alone digitized [REF and rephrase and check these stats \cite{Ari_o_2010}]. This lack of documentation makes it easy to underestimate not just the scale of NHM collections, but also the amount of collaborative work  - often extremely dispersed in both time and space - that supports their preservation. NHM workers hold a particularly long view of information preservation, collaboration and curation -- one which could and should fundamentally inform best practices in data curation, as well as design practices for distributed, cooperative work.

add some refs about how this work uses very long lived data sources and has ver long citation half-lives

maybe add in a sentence clarifying relationship to biodiversity/taxonomic informatics projects

\begin{quote}
“To sum up, it is proposed that most users should interact with a relational model of the data consisting of a collection of time-varying relationships (rather than relations).”
\end{quote}