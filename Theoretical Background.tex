\section{Theoretical Background}
\subsection{Codd}
Most contemporary databases follow E. F. Codd's relational data model, which was meant to address problems of “shared” data banks, by  “protecting users from having to know how data is organized” in a computer system  \cite[p. 377]{Codd_1970}. Going against the tree structures and network data models popular in the 1970's, Codd conceived of a set theoretic approach to data representation. This was later formalized as the ANSI-SPARC three level architecture, which separates a data bank's structure into three distinct levels: 1. Physical levels describing how data are stored; 2. Logical levels describing what data are stored, and; 3. Views levels providing access that provide access to stored data, but hide the details of the physical and logical levels from an end-user \cite{tsichritzis1978ansi}. In hiding these details, Codd's relational model emphasizes usability, as he wrote “there is one consideration of \emph{absolutely paramount importance} – and that is the convenience of the majority of users” \cite[p.2]{Codd_1971}.

In Codd's relational model, the separation of a database's architecture also results in a division of labor between designers and administrators at the logical level, and end-users at the views level. In practice, these divisions are not based on rigid types, but rather roles that different users might play in interacting with a database at different points in time. For instance, in a natural history museum an individual might help design and administer a database that she also uses as a part of her own research. This is especially common in information-rich-but-personnel-poor workplaces, where desktop applications such as Microsoft Access and FileMaker Pro reduce the burden of having to be conversant in set theory just to create reliable data storage. 

A few CSCW and HCI studies have described the problematic nature of these collapsed database roles, including: Batra's work on error types in database design \cite{Batra_1993} which are described in terms of  four 'gulfs' that emerge between context of use and design; Olson's description of the organizational super-administrator \cite{olson2010} database, and Dourish and Edwards' design of a toolkit to reduce the burden of "locking" style commitments into software code or design constraints of a database schema \cite{Dourish_2000}. Also important to note is Jagadish et al's study of biological database usability, in which they note the importance of presentation data models (or views) of data \cite{jagadish2007making}. Our work here focuses on how the division between logical and views levels of database architecture impact the collaborative work that can be facilitated by a relational database. We emphasize that rather than impeding work, the rigidness of the relational model spurs innovation that leads to adaptive database design. Supporting those design activities is the major goal of this work. 

\subsection{Brand}

One of the difficulties in supporting database and logical schema design work is rooted in the difficulty of talking about database and logical schema design. Evaluation of logical models gets mathematical fast, and their formal nature can hinder design work. We turn to architecture -- specifically Stewart Brand’s analysis of buildings through time \cite{brand1995buildings} -- as an metaphorical framework that can help illustrate the character of database evolution without requiring a primer on relational algebra. Brand contrasts the initial appearance and architect’s vision of a building with the actual lived experience of the people inhabiting the building. He continues this analysis by exploring use over years, decades and even centuries noting how the requirements of the inhabitants of the building change, and so they change the structure and the meaning of parts of the building; adding, demolishing and modifying. Brand’s ideas have been found useful in the context of an HCI perspective on ubiquitous technologies for domestic environments \cite{rodden2003evolution}. We believe that his analysis of buildings can also serve as a productive metaphor for exploring the way that databases change over similarly long time periods, particularly for collaborative environments of database use, such as natural history museums \cite{thomer2014databases}. 

\subsection{The nature of natural history work}

We have selected natural history museums (NHMs) as the focus for this research as a result of our own experience designing databases, and collaborating within natural history research settings. NHM work, particularly that done by collections managers (CMs)\footnote{In NHMs, it is the collection manager that is typically in charge of the the actual day-to-day maintenance of collections and databases -- not the curator, who typically does less curation in the sense of physically caring for specimens and their data, but rather functions more like a researcher administrator: writing papers and grants, and making policy decisions.}, is deeply sociotechnical: it involves mediating between physical specimen collections (these can range from fossils to sea shells to whole organisms preserved in alcohol), paper records, and many information communication technologies, including relational databases. It also involves coordination between volunteers, curators, IT staff, visiting researchers, and other departments. Thus, any given day’s work will involve some combination of wrangling both people and information structures. In this research, our participants estimated that they spent anywhere from 40-80\% of their time working with databases, or supervising the work of volunteers working with databases.

Data stored in electronic collections databases is rarely created \textit{de novo}; rather, these databases are better viewed as the latest manifestation of a hundreds-year-old tradition of NHM record keeping. The data they contain have been "digitized" (often through transcription) from paper ledgers, card catalogs, and individual labels. Many NHMs were early adopters of database technology, and it is not unusual for catalogs from the 1970s and 80s to have been stored on punch-cards. As a result, many of these systems were migrated during the 1990's to custom built, but proprietary, systems such as dBase, Paradox and Access. Since the early 2000s, NHMs have been implementing community developed database designs, such as Arctos, Specify and KE EMu, that pre-define schemas specifically for use with natural history data. The move to community-developed databases is paralleled by a move toward community-developed data standards and schemas. However, while many collections include similar information in their records – for instance, a catalog number, a specimen’s scientific name, a description of the locality from which it was collected – the way in which this data is recorded is often quite idiosyncratic to an institution’s local cataloging rules and practices. And as a result, collections management databases often have correspondingly idiosyncratic legacy data structures.

Ideally, collections databases would act as both a finding aid for the physical collection, and a research dataset in and of themselves (see \cite{Chapman2005} for a broad review of research uses). This dual role would include listing the name and catalog number of each specimen, as well as additional contextualizing data such as the name of its original collector and the location from which it was collected. At best, these databases are partial representations of the collection -- and, as Bowker notes, the natural world \cite{Bowker_2000}. The vast majority of museum collections are \textit{not} fully cataloged; and many of the catalogs that do exists are still in need of digitization \cite{Beaman_2012}. Of the estimated 1.2 \textit{billion} specimens worldwide, only 9\% have been cataloged, let alone digitized \cite{Ari_o_2010}. This lack of documentation makes it easy to underestimate not just the scale of NHM collections, but also the amount of collaborative work -- often extremely dispersed in both time and space -- that supports their preservation. And as a result, NHM workers hold a particularly long view of information preservation, collaboration, and curation.