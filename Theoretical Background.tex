\section{Theoretical Background}

\subsection{Codd}
\begin{itemize}
\item Tripartite model (ANSI)
\item Logical modeling as a CSCW issue
\begin{itemize}
\item Relational model as designed for distinct user/administrator roles

\begin{quote}
“To sum up, it is proposed that most users should interact with a relational model of the data consisting of a collection of time-varying relationships (rather than relations).”
\end{quote}

All relational databases owe their logical underpinnings to E. F. Codd’s “Relational Model of Data for Large Shared Data Banks” \cite{Codd_1970}. In this canonical paper, Codd describes a then-new way of organizing data: not as a graph or network, but rather in a set theoretic way as a series of tables (relations) connected by relationships.  

Codd was as concerned with supporting cooperative work as he was with ensuring data independence: he explicitly intended to address problems of “shared” data banks, by  “[protecting] users from having to know how data is organized” in a computer system (pp. 377, \cite{Codd_1970}; in a later paper he declares that in the choice of data structures,  “there is one consideration of absolutely paramount importance – and that is the convenience of the majority of users” (1971, pp.2; emphasis his). 

Codd’s user-centered design goals, however, do not seem to have translated to many modern instantiations of relation databases. For one thing, Codd’s notion that users “must be protected” from data structures is rooted in an assumption that has since become outdated: that databases are operated by two clear-cut groups of people, users and administrators.  While a clear division of labor between administrators and users surely exists in some organizations – and was likely more usual at the time of Codd’s writing – the development and continued prevalence of desktop database applications such as Microsoft Access and FileMaker Pro has been blurring (if not erasing) the line between database user and administrator in information-rich-but-personnel-poor workplaces for decades (if only because many of these databases do not exist at the scale necessary to render such a distinction possible or financially viable). 
Relational databases’ “roles” have shifted over time as well. Many early conceptions of relational databases in particular were with business information needs in mind; Codd uses the example of a “parts” database for a machine warehouse, and many database design textbooks and courses continue to use similar versions of this use case today.  But the continued resurrection of this transactional use case ignores the more complex existence that relational databases have in a research setting, in which they are part infrastructure for information storage and retrieval, part research objects in and of themselves, and wholly what Bietz and Lee call “boundary negotiating objects” (2009): artifacts in which research methods and worldviews are negotiated into being, and crystallized into canon.  

\item Lots of discussion of need for data independence between physical/logical levels – what about logical/conceptual? How do those interact?
\item And what about logical structure and day-to-day work practices? How do those interact?
\item Note related work in HCI/CSCW - Batra 1993; Topi & Ramish 2003; more
\end{itemize}
\end{itemize}

\subsection{Brand}

\begin{itemize}
\item Brand gives us a framework to start answering those questions
\item	it's unintuitive that a building learns from its inhabitants and vice versa’ Equally unintuitive is that conceptual and logical levels learn form one another (e.g. that databases and people learn from each other)
\item Brand’s 6 S’s as they translate to databases
\item Include figure showing table growth over time
\end{itemize}

\subsection{Related Work in CSCW}

[depending on what isn't folded into the sections above and below]

\subsection{The nature of natural history work}

We have selected natural history museums as the focus for this case study because they are an excellent "model organism" for [the kind of database work we will have presumably described above]. NHM work -- particularly, that done by collections managers (CMs) -- is deeply technosocial: it involves mediating between physical specimen collections (these can range from fossils to sea shells to whole organisms preserved in alcohol, depending on the discipline), paper records, relational databases and anticipated future platforms; and between volunteers, curators\footnote{we note that in NHMs, the "curator" tends to do less curation in the sense of physically caring for specimens and their data, but rather functions more like a researcher administrator: writing papers and grants, and making policy decisions. Collections managers are typically in charge of the the actual day-to-day maintenance of collections and databases.}, IT staff, visiting researchers, and other departments. Thus, any given day’s work will involve some combination of wrangling both people and information structures. Indeed, each CM we spoke with for this research spends anywhere from 40-80\% of her time working with databases (typically closer to the latter) or supervising/checking the work of volunteers working with databases.

The data in electronic collections databases is rarely created \textit{de novo}; rather, these databases are better viewed as the latest manifestation of a hundreds-year-old tradition of NHM record keeping. The data they contain have been "digitized" (transcribed) from paper ledgers, card catalogs, and individual labels. NHM's have been somewhat early adopters of databases technology, and many of the existing electronic catalogs were originally created via punch card technology in the 1970s and 80s; in the 90s many of these systems were migrated to custom built, but proprietary, systems such as dBase, Paradox and Access; and since the early 200s, more and more museums have been using "off-the-shelf" databases with pre-defined data schemas designed specifically for NHMs, such as Arctos, Specify and KE EMu. The move to community-developed databases such as Arctos and Specify is paralleled by a move toward community-developed data standards and schemas.  However, while many collections include similar information in their records – for instance, a catalog number, a specimen’s scientific name, a description of the locality from which it was collected – the way in which this data is recorded is often quite idiosyncratic to an institution’s local cataloging rules and practices, as well as to the domain of study; as a result, collections management databases often have correspondingly idiosyncratic legacy data structures.

Ideally, collections databases would act as both a finding aid for the physical collection, and a research dataset in and of themselves (see \cite{Chapman2005} for a broad review of research uses): listing the name and catalog number of each specimen, as well as additional contextualizing data such as the name of its original collector and the location from which it was collected. However, these databases are partial representations of the collection -- and, as Bowker notes, the natural world \cite{Bowker_2000} -- at best. The vast majority of museum collections are \textit{not} fully cataloged; and many of the catalogs that do exists are still in need of digitization \cite{Beaman_2012}. Of the estimated 1.2 billion specimens worldwide, only [number] 9\% have been cataloged, let alone digitized [REF and rephrase and check these stats \cite{Ari_o_2010}]. This lack of documentation makes it easy to underestimate not just the scale of NHM collections, but also the amount of collaborative work  - often extremely dispersed in both time and space - that supports their preservation. NHM workers hold a particularly long view of information preservation, collaboration and curation -- one which could and should fundamentally inform best practices in data curation, as well as design practices for distributed, cooperative work.

add some refs about how this work uses very long lived data sources and has ver long citation half-lives

maybe add in a sentence clarifying relationship to biodiversity/taxonomic informatics projects
