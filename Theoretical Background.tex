\section{Theoretical Background}
Most contemporary databases follow E. F. Codd's relational data model, which was meant to address problems of “shared” data banks, by  “protecting users from having to know how data is organized” in a computer system  \cite[p. 377]{Codd_1970}. Going against the tree-structures or network data models popular in the 1970's, Codd conceived of a set theoretic approach to the representing data by separating a system's architecture into three distinct levels: 1. Physical levels describe how data are stored; 2. Logical levels describe what data are stored, and; 3. Views levels provide access to stored data, but hide details of the physical and logical levels from an end-user. In hiding these details, Codd's relational model emphasizes usability, as he wrote “there is one consideration of \emph{absolutely paramount importance} – and that is the convenience of the majority of users” \cite[p.2; emphasis his]{Codd_1971}.

In Codd's relational model, the division of a database's architecture also results in a division of labor between designers and administrators at the logical level, and end-users at the views level. In practice, these divisions are not based on rigid types, but rather roles that different people play in interacting with a database at different points in time. For instance, in a natural history museum an individual might help design and administer a database that she also uses as a part of her own research. This is especially common in information-rich-but-personnel-poor workplaces, where desktop applications such as Microsoft Access and FileMaker Pro reduce the burden of having to learn set-theory just to create a relational database. A few CSCW and HCI studies have described the problematic nature of these collapsed database roles, including: Batra's work on error types in database design (1993) which are described in terms of  four 'gulfs' that emerge between context of use and design; Olson's description of the organizational super-administrator (2010), and ; Dourish and Edwards design of a toolkit to reduce the burden of "locking" style commitments into software code or design constraints of a database schema (2000). Our work here focuses on how the division between logical and views levels of database architecture impact the collaborative work that can be facilitated by a relational database. We emphasize that rather than impeding work, the rigidness of the relational model spurs innovation that leads to adaptive database design. Supporting those design activities is the major goal of this work. 

\subsection{Brand}

Brand’s analysis of buildings \cite{brand1995buildings}, highlights a number of important issues that arise when taking a longer term perspective. He contrasts the initial appearance of a building and the architect’s vision (often the major criteria used in determining awards), with the actual use experience of the people living in the building. He continues this analysis by exploring use over years, decades and even centuries noting how the requirements of the inhabitants of the building change and so they change the structure and the meaning of parts of the building; adding, demolishing and modifying. Over longer time periods, a building may have different inhabitants and be used for changing purposes. Brand highlighted six different shearing layers of change in buildings: Site, Structure, Skin, Services, Space Plan, and Stuff. These layers change at very different rates - from days to centuries - and with various levels of ease and cost. Brand’s ideas have been found useful in the context of an HCI perspective on ubiquitous technologies for domestic environments \cite{rodden2003evolution}. We believe that his analysis of buildings can also serve as a productive metaphor for exploring the way that databases change over similarly long time periods 
 \cite{thomer2014databases}. 

\subsection{The nature of natural history work}

We have selected natural history museums (NHMs) as the focus for this research as a result of our own experience designing databases, and collaborating within natural history research settings. NHMs are often sites where collaboration is necessary for making substantive contributions to a natural history research agenda, and they suffer from chronic under-funding and staffing issues- in other words,  they represent a local that we can expect innovation, and ingenuity out of necessity. NHM work -- particularly, that done by collections managers (CMs)\footnote{In NHMs, it is the collection manager that is typically in charge of the the actual day-to-day maintenance of collections and databases -- not the curator, who typically does less curation in the sense of physically caring for specimens and their data, but rather functions more like a researcher administrator: writing papers and grants, and making policy decisions.} -- is deeply sociotechnical: it involves mediating between physical specimen collections (these can range from fossils to sea shells to whole organisms preserved in alcohol, depending on the discipline), paper records, relational databases and anticipated future platforms; and between volunteers, curators, IT staff, visiting researchers, and other departments. Thus, any given day’s work will involve some combination of wrangling both people and information structures. Indeed, each CM we spoke with for this research spends anywhere from 40-80\% of her time working with databases (typically closer to the latter) or supervising/checking the work of volunteers working with databases.

The data in electronic collections databases is rarely created \textit{de novo}; rather, these databases are better viewed as the latest manifestation of a hundreds-year-old tradition of NHM record keeping. The data they contain have been "digitized" (transcribed) from paper ledgers, card catalogs, and individual labels. NHM's have been somewhat early adopters of database technology, and many of the existing electronic catalogs were originally created via punch card technology in the 1970s and 80s; in the 90s many of these systems were migrated to custom built, but proprietary, systems such as dBase, Paradox and Access; and since the early 200s, more and more museums have been using "off-the-shelf" databases with pre-defined data schemas designed specifically for NHMs, such as Arctos, Specify and KE EMu. The move to community-developed databases such as Arctos and Specify is paralleled by a move toward community-developed data standards and schemas.  However, while many collections include similar information in their records – for instance, a catalog number, a specimen’s scientific name, a description of the locality from which it was collected – the way in which this data is recorded is often quite idiosyncratic to an institution’s local cataloging rules and practices, as well as to the domain of study; as a result, collections management databases often have correspondingly idiosyncratic legacy data structures.

Ideally, collections databases would act as both a finding aid for the physical collection, and a research dataset in and of themselves (see \cite{Chapman2005} for a broad review of research uses): listing the name and catalog number of each specimen, as well as additional contextualizing data such as the name of its original collector and the location from which it was collected. However, these databases are partial representations of the collection -- and, as Bowker notes, the natural world \cite{Bowker_2000} -- at best. The vast majority of museum collections are \textit{not} fully cataloged; and many of the catalogs that do exists are still in need of digitization \cite{Beaman_2012}. Of the estimated 1.2 billion specimens worldwide, only [number] 9\% have been cataloged, let alone digitized [REF and rephrase and check these stats \cite{Ari_o_2010}]. This lack of documentation makes it easy to underestimate not just the scale of NHM collections, but also the amount of collaborative work  - often extremely dispersed in both time and space - that supports their preservation. NHM workers hold a particularly long view of information preservation, collaboration and curation -- one which could and should fundamentally inform best practices in data curation, as well as design practices for distributed, cooperative work.

add some refs about how this work uses very long lived data sources and has ver long citation half-lives

maybe add in a sentence clarifying relationship to biodiversity/taxonomic informatics projects

\begin{quote}
“To sum up, it is proposed that most users should interact with a relational model of the data consisting of a collection of time-varying relationships (rather than relations).”
\end{quote}