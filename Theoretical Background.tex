\section{Theoretical Background}

\subsection{Codd}
\begin{itemize}
\item Tripartite model (ANSI)
\item Logical modeling as a CSCW issue
\begin{itemize}
\item Relational model as designed for distinct user/administrator roles
\item Lots of discussion of need for data independence between physical/logical levels – what about logical/conceptual? How do those interact?
\item And what about logical structure and day-to-day work practices? How do those interact?
\item Note related work in HCI/CSCW - Batra 1993; Topi & Ramish 2003; more
\end{itemize}
\end{itemize}


\subsection{Brand}

\begin{itemize}
\item Brand gives us a framework to start answering those questions
\item	it's unintuitive that a building learns from its inhabitants and vice versa’ Equally unintuitive is that conceptual and logical levels learn form one another (e.g. that databases and people learn from each other)
\item Brand’s 6 S’s as they translate to databases
\item Include figure showing table growth over time
\end{itemize}

\subsection{Related Work in CSCW}


\subsection{The nature of natural history work}

We have chosen natural history museums as the focus for this case study because they are an excellent "model organism" for [the kind of database work we will have presumably described above]. NHM work -- particularly, collections management -- is deeply technosocial: it involves mediating between physical specimen collections (these can range from fossils to sea shells to whole organisms preserved in alcohol, depending on the discipline), paper records, relational databases and anticipated future platforms; and between volunteers, IT staff, curators\footnote{we note that in NHMs, the "curator" tends to do less curation in the sense of physically caring for specimens and their data, but rather functions more like a researcher administrator: writing papers and grants, and making policy decisions. Collections managers are typically in charge of the the actual day-to-day maintenance of collections and databases.}, visiting researchers, and other departments. It also involves huge amounts of collaboration across both time and space: collections databases are often decades years old; and really, hundreds, if paper catalog ledgers are included!) and linked with researchers and collections worldwide. Thus, any given day’s work will involve some combination of wrangling both people and information structures. Indeed, each CM we spoke with for this research spends anywhere from 40-80\% of her time working with databases (typically closer to the latter) or supervising/checking the work of volunteers working with databases.

These are primarily used for the storage, retrieval, and curation of data describing specimens in the collection.

Prior to the use of computers, these records would have been stored in paper catalog ledgers, card catalogs, or on labels affixed to jars or drawers; several of our participants noted that they still directly handwrite data into paper catalog ledgers, which is later transcribed into the database; even once these ledgers are digitized, they’re still kept for provenance, and as a backup of the electronic database. While many collections include similar information in their records – for instance, a catalog number, a specimen’s scientific name, a description of the locality from which it was collected – the way in which this data is recorded is often quite idiosyncratic to an institution’s local cataloging rules and practices, as well as to the domain of study; as a result, collections management databases often have correspondingly idiosyncratic legacy data structures.

Thus, while there are now a number of database systems designed specifically for NHMs

(NHMs have been surprisingly early adopters of computer technology in many ways)
collections management databases aren’t created and maintained for their own sake; they’re first and foremost a tool to facilitate the management and use of physical specimen collections. 
Dual meanings of Digitization – transcription + imaging


record keeping practies in museums and legacy of early adoption of computers
what do we mean when we say digitization
relationship to biodiversity/taxonomic informatics projects

